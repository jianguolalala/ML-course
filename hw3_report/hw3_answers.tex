%
%
\documentclass[12pt,twoside]{article}

\input{macros}

\usepackage{amsmath}
\usepackage{url}
\usepackage{mdwlist}
\usepackage{graphicx}
\usepackage{clrscode3e}
\newcommand{\isnotequal}{\mathrel{\scalebox{0.8}[1]{!}\hspace*{1pt}\scalebox{0.8}[1]{=}}}
\usepackage{listings}
\usepackage{tikz}
\usepackage{float}
\usetikzlibrary{arrows}
\usetikzlibrary{matrix}
\usetikzlibrary{positioning}
\usetikzlibrary{shapes.geometric}
\usetikzlibrary{shapes.misc}
\usetikzlibrary{trees}

\usepackage{hyperref}
\usepackage[all]{hypcap}
\usepackage{caption}
\usepackage{subfigure}
\captionsetup{hypcap=true}

\newcommand{\answer}{
 \par\medskip
 \textbf{Answer:}
}

\newcommand{\collaborators}{ \textbf{Collaborators:}
%%% COLLABORATORS START %%%

\tabT Name: Wang Yanwei

\tabT Student ID: 11821049
%%% COLLABORATORS END %%%
}

\newcommand{\answerIa}{ \answer
%%% PROBLEM 1(a) ANSWER START %%%
Test accuracy is 0.924.
%%% PROBLEM 1(a) ANSWER END %%%
}

\newcommand{\answerIIa}{ \answer
%%% PROBLEM 2(a) ANSWER START %%%
\begin{figure}[H]
	\centering
	\includegraphics[scale=0.7]{k1.png}
	\caption{K = 1.}
\end{figure}
\begin{figure}[H]
	\centering
	\includegraphics[scale=0.7]{k10.png}
	\caption{K = 10.}
\end{figure}
\begin{figure}[H]
	\centering
	\includegraphics[scale=0.7]{k100.png}
	\caption{K = 100.}
\end{figure}
%%% PROBLEM 2(a) ANSWER END %%%
}


\newcommand{\answerIIb}{ \answer
%%% PROBLEM 2(b) ANSWER START %%%
Use dimension reduction method to view data in low dimension, choose a proper K according to the data struct.
%%% PROBLEM 2(b) ANSWER END %%%
}

\newcommand{\answerIIc}{ \answer
	%%% PROBLEM 2(c) ANSWER START %%%
See the code.
	%%% PROBLEM 2(c) ANSWER END %%%
}



\newcommand{\answerIIIa}{ \answer 
%%% PROBLEM 3(a) ANSWER START %%%
\begin{figure}[H]
	\centering
	\includegraphics[scale=0.7]{dt.png}
	\caption{Desion Tree.}
\end{figure}
%%% PROBLEM 3(a) ANSWER END %%%

}

\newcommand{\answerIIIIa}{ \answer
	%%% PROBLEM 4(a) ANSWER START %%%
	
\begin{figure}[H]
	\centering
	\includegraphics[scale=0.7]{4a_min.png}
	\caption{Kmeans Min SD.}
\end{figure}
\begin{figure}[H]
	\centering
	\includegraphics[scale=0.7]{4a_max.png}
	\caption{Kmeans Max SD.}
\end{figure}
	%%% PROBLEM 4(a) ANSWER END %%%
}

\newcommand{\answerIIIIb}{ \answer
	%%% PROBLEM 4(b) ANSWER START %%%
	Run code several times, choose the best model which has the smallest SD.
	
	%%% PROBLEM 4(b) ANSWER END %%%
}

\newcommand{\answerIIIIc}{ \answer
	%%% PROBLEM 4(c) ANSWER START %%%
\begin{figure}[H]
	\centering
	\includegraphics[scale=0.7]{4k10.png}
	\caption{K = 10.}
\end{figure}
\begin{figure}[H]
	\centering
	\includegraphics[scale=0.7]{4k20.png}
	\caption{K = 20.}
\end{figure}
\begin{figure}[H]
	\centering
	\includegraphics[scale=0.7]{4k50.png}
	\caption{K = 50.}
\end{figure}
	%%% PROBLEM 4(c) ANSWER END %%%
}

\newcommand{\answerIIIId}{ \answer
	%%% PROBLEM 4(d) ANSWER START %%%
	\begin{figure}[H]
		\centering
		\includegraphics[scale=0.7]{vq64.png}
		\caption{vector quantization.}
	\end{figure}
The compress ration is 0.75 , I ignore 64 $\times$24 bits data, beacuse it is rather small to the $H\times W \times 24$ image data.
	%%% PROBLEM 4(d) ANSWER END %%%
}

\setlength{\oddsidemargin}{0pt}
\setlength{\evensidemargin}{0pt}
\setlength{\textwidth}{6.5in}
\setlength{\topmargin}{0in}
\setlength{\textheight}{8.5in}

% Fill these in!
\newcommand{\theproblemsetnum}{3}
\newcommand{\releasedate}{Oct 15, 2019}
\newcommand{\partaduedate}{Oct 31, 2019}
\newcommand{\tabUnit}{3ex}
\newcommand{\tabT}{\hspace*{\tabUnit}}

\begin{document}

\handout{Homework \theproblemsetnum}{\releasedate}

%\textbf{Both theory and programming questions} are due {\bf \partaduedate} at
%{\bf 11:59PM}.
%
\collaborators
%Please download the .zip archive for this problem set.
% Your grade will be based on both your solutions and the grading explanation.

\medskip

\hrulefill

\begin{problems}

\problem \textbf{Neural Networks}

In this problem, we will implement the feedforward and backpropagation process of
the neural networks.
\begin{problemparts}
\problempart 
\answerIa

\end{problemparts}

\problem \textbf{K-Nearest Neighbor}

In this problem, we will play with K-Nearest Neighbor (KNN) algorithm and try it on
real-world data. Implement KNN algorithm (in \emph{knn.m}/\emph{knn.py}), then answer the following questions.
\begin{problemparts}
\problempart
Try KNN with different K and plot the decision boundary.


\answerIIa

\problempart We have seen the effects of different choices of K. How can you choose a proper K
when dealing with real-world data ?

\answerIIb

\problempart Finish \emph{hack.m}/\emph{hack.py} to recognize the CAPTCHA image using KNN algorithm.

\answerIIc
\end{problemparts}

\problem \textbf{Decision Tree and ID3}

Consider the scholarship evaluation problem: selecting scholarship recipients based on gender
and GPA. Given the following training data:

\answerIIIa


\problem \textbf{K-Means Clustering}

Finally, we will run our first unsupervised algorithm – k-means clustering.
\begin{problemparts}
	\problempart
	Visualize the process of k-means algorithm for the two trials.
	
	\answerIIIIa
	
	\problempart How can we get a stable result using k-means?
	
	\answerIIIIb
	
	\problempart  Visualize the centroids.
	
	\answerIIIIc
	
	\problempart  Vector quantization.
	
	\answerIIIId
	
\end{problemparts}


\end{problems}
\end{document}
