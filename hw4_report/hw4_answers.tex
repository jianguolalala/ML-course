
\documentclass[12pt,twoside]{article}

\input{macros}

\usepackage{amsmath}
\usepackage{url}
\usepackage{mdwlist}
\usepackage{graphicx}
\usepackage{clrscode3e}
\newcommand{\isnotequal}{\mathrel{\scalebox{0.8}[1]{!}\hspace*{1pt}\scalebox{0.8}[1]{=}}}
\usepackage{listings}
\usepackage{tikz}
\usepackage{float}
\usetikzlibrary{arrows}
\usetikzlibrary{matrix}
\usetikzlibrary{positioning}
\usetikzlibrary{shapes.geometric}
\usetikzlibrary{shapes.misc}
\usetikzlibrary{trees}

\usepackage{hyperref}
\usepackage[all]{hypcap}
\usepackage{caption}
\usepackage{subfigure}
\captionsetup{hypcap=true}

\newcommand{\answer}{
 \par\medskip
 \textbf{Answer:}
}

\newcommand{\collaborators}{ \textbf{Collaborators:}
%%% COLLABORATORS START %%%

\tabT Name: Wang Yanwei

\tabT Student ID: 11821049
%%% COLLABORATORS END %%%
}

\newcommand{\answerIa}{ \answer
%%% PROBLEM 1(a) ANSWER START %%%
Abviously, Spectral Cluster can classiify the data correctly, but Kmeans not.

\begin{figure}[H]
	\centering
	\subfigure[spectral cluster]{\includegraphics[scale=0.5]{spec.png}}
	\subfigure[kmeans]{\includegraphics[scale=0.5]{kmeans.png}}	
	\caption{spectral cluster VS kmeans}
\end{figure}
%%% PROBLEM 1(a) ANSWER END %%%
}

\newcommand{\answerIb}{ \answer
	%%% PROBLEM 1(b) ANSWER START %%%
	 
	Spectral Clustering: Accuracy=0.5782410917361638, mutual information=0.6684473568298703 
	
	kmeans: Accuracy=0.5134950720242608, mutual information=0.36631839357492624
	
	%%% PROBLEM 1(b) ANSWER END %%%
}
\newcommand{\answerIIa}{ \answer
%%% PROBLEM 2(a) ANSWER START %%%
\begin{figure}[h]
	\centering
	\includegraphics[scale=1.]{res.png}
	\caption{rotated image}
\end{figure}
%%% PROBLEM 2(a) ANSWER END %%%
}


\newcommand{\answerIIb}{ \answer
%%% PROBLEM 2(b) ANSWER START %%%
 
\begin{figure}[H]
	\includegraphics[scale=0.7]{eigenface.png}
	\caption{eigen face}
\end{figure}

When low dimensional number is 8, accuracy is 0.745 \\
When low dimensional number is 16, accuracy is 0.825 \\
When low dimensional number is 32, accuracy is 0.845 \\
When low dimensional number is 64, accuracy is 0.865 \\
When low dimensional number is 128, accuracy is 0.865 \\
 \\

\begin{figure}[H]
	\includegraphics[scale=0.7]{128.png}
	\caption{128 dim}
\end{figure}
%%% PROBLEM 2(b) ANSWER END %%%
}



\setlength{\oddsidemargin}{0pt}
\setlength{\evensidemargin}{0pt}
\setlength{\textwidth}{6.5in}
\setlength{\topmargin}{0in}
\setlength{\textheight}{8.5in}

% Fill these in!
\newcommand{\theproblemsetnum}{4}
\newcommand{\releasedate}{Oct 29, 2019}
\newcommand{\partaduedate}{Nov 09, 2019}
\newcommand{\tabUnit}{3ex}
\newcommand{\tabT}{\hspace*{\tabUnit}}

\begin{document}

\handout{Homework \theproblemsetnum}{\releasedate}

%\textbf{Both theory and programming questions} are due {\bf \partaduedate} at
%{\bf 11:59PM}.
%
\collaborators
%Please download the .zip archive for this problem set, and refer to the
%\texttt{README.txt} file for instructions on preparing your solutions.
%
%We will provide the solutions to the problem set 10 hours after the problem set
%is due. You will have to read the solutions, and write a brief \textbf{grading
%explanation} to help your grader understand your write-up. You will need to
%submit the grading explanation by \textbf{Thursday, November 3rd, 11:59PM}. Your
%grade will be based on both your solutions and the grading explanation.

\medskip

\hrulefill

\begin{problems}

\problem \textbf{Spectral Clustering}

In this problem, we will try a dimensionality reduction based clustering algorithm – Spectral
Clustering.
\begin{problemparts}
\problempart 
We will first experiment Spectral Clustering on synthesis data
\answerIa

\problempart
Now let us try Spectral Clustering on real-world data.
\answerIb

\end{problemparts}

\problem \textbf{Principal Component Analysis}
 Let us deepen our understanding of PCA by the following problems.
\begin{problemparts}
\problempart
Your task is to implement \textit{hack\_pca.m} to recover the rotated CAPTCHA image using PCA.


\answerIIa

\problempart Now let us apply PCA to a face image dataset.

\answerIIb


\end{problemparts}



\end{problems}
\end{document}
